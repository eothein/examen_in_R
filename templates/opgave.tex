\documentclass{exam}

%% packages
\usepackage{a4wide,color,verbatim,Sweave,url,xargs,amsmath,hyperref,booktabs,xcolor,tabulary}
\usepackage[utf8]{inputenc}
\usepackage[dutch]{babel}

%% new environments
\newenvironment{question}{\item}{}
\newenvironment{answerlist}{\renewcommand{\labelenumi}{(\alph{enumi})}\begin{enumerate}}{\end{enumerate}}


%% compatibility with pandoc
\providecommand{\tightlist}{\setlength{\itemsep}{0pt}\setlength{\parskip}{0pt}}
\setkeys{Gin}{keepaspectratio}

%% Generating the command for styling the questions
\newcommand{\extext}[1]{\phantom{\large #1}}
\newcommandx{\exmchoice}[9][2=-,3=-,4=-,5=-,6=-,7=-,8=-,9=-]{%
	\mbox{(a) \,\, \framebox[8mm]{\rule[-1mm]{0mm}{5mm} \hspace*{-1.6mm} \extext{#1}} \hspace*{2mm}}%
	\if #2- \else \mbox{(b) \,\, \framebox[8mm]{\rule[-1mm]{0mm}{5mm} \hspace*{-1.6mm} \extext{#2}} \hspace*{2mm}} \fi%
	\if #3- \else \mbox{(c) \,\, \framebox[8mm]{\rule[-1mm]{0mm}{5mm} \hspace*{-1.6mm} \extext{#3}} \hspace*{2mm}} \fi%
	\if #4- \else \mbox{(d) \,\, \framebox[8mm]{\rule[-1mm]{0mm}{5mm} \hspace*{-1.6mm} \extext{#4}} \hspace*{2mm}} \fi%
	\if #5- \else \mbox{(e) \,\, \framebox[8mm]{\rule[-1mm]{0mm}{5mm} \hspace*{-1.6mm} \extext{#5}} \hspace*{2mm}} \fi%
	\if #6- \else \mbox{(f) \,\, \framebox[8mm]{\rule[-1mm]{0mm}{5mm} \hspace*{-1.6mm} \extext{#6}} \hspace*{2mm}} \fi%
	\if #7- \else \mbox{(g) \,\, \framebox[8mm]{\rule[-1mm]{0mm}{5mm} \hspace*{-1.6mm} \extext{#7}} \hspace*{2mm}} \fi%
	\if #8- \else \mbox{(h) \,\, \framebox[8mm]{\rule[-1mm]{0mm}{5mm} \hspace*{-1.6mm} \extext{#8}} \hspace*{2mm}} \fi%
	\if #9- \else \mbox{(i) \,\, \framebox[8mm]{\rule[-1mm]{0mm}{5mm} \hspace*{-1.6mm} \extext{#9}} \hspace*{2mm}} \fi%
}
\newcommandx{\exclozechoice}[9][2=-,3=-,4=-,5=-,6=-,7=-,8=-,9=-]{\setcounter{enumiii}{1}%
	\mbox{\roman{enumiii}. \, \framebox[8mm]{\rule[-1mm]{0mm}{5mm} \hspace*{-1.6mm} \extext{#1}} \hspace*{2mm}\stepcounter{enumiii}}%
	\if #2- \else \mbox{\roman{enumiii}. \, \framebox[8mm]{\rule[-1mm]{0mm}{5mm} \hspace*{-1.6mm} \extext{#2}} \hspace*{2mm}\stepcounter{enumiii}} \fi%
	\if #3- \else \mbox{\roman{enumiii}. \, \framebox[8mm]{\rule[-1mm]{0mm}{5mm} \hspace*{-1.6mm} \extext{#3}} \hspace*{2mm}\stepcounter{enumiii}} \fi%
	\if #4- \else \mbox{\roman{enumiii}. \, \framebox[8mm]{\rule[-1mm]{0mm}{5mm} \hspace*{-1.6mm} \extext{#4}} \hspace*{2mm}\stepcounter{enumiii}} \fi%
	\if #5- \else \mbox{\roman{enumiii}. \, \framebox[8mm]{\rule[-1mm]{0mm}{5mm} \hspace*{-1.6mm} \extext{#5}} \hspace*{2mm}\stepcounter{enumiii}} \fi%
	\if #6- \else \mbox{\roman{enumiii}. \, \framebox[8mm]{\rule[-1mm]{0mm}{5mm} \hspace*{-1.6mm} \extext{#6}} \hspace*{2mm}\stepcounter{enumiii}} \fi%
	\if #7- \else \mbox{\roman{enumiii}. \, \framebox[8mm]{\rule[-1mm]{0mm}{5mm} \hspace*{-1.6mm} \extext{#7}} \hspace*{2mm}\stepcounter{enumiii}} \fi%
	\if #8- \else \mbox{\roman{enumiii}. \, \framebox[8mm]{\rule[-1mm]{0mm}{5mm} \hspace*{-1.6mm} \extext{#8}} \hspace*{2mm}\stepcounter{enumiii}} \fi%
	\if #9- \else \mbox{\roman{enumiii}. \, \framebox[8mm]{\rule[-1mm]{0mm}{5mm} \hspace*{-1.6mm} \extext{#9}} \hspace*{2mm}} \fi%
}
\newcommand{\exnum}[9]{%
	\mbox{\framebox[8mm]{\rule[-1mm]{0mm}{5mm} \hspace*{-1.6mm} \extext{#1}}}%
	\mbox{\framebox[8mm]{\rule[-1mm]{0mm}{5mm} \hspace*{-1.6mm} \extext{#2}}}%
	\mbox{\framebox[8mm]{\rule[-1mm]{0mm}{5mm} \hspace*{-1.6mm} \extext{#3}}}%
	\mbox{\framebox[8mm]{\rule[-1mm]{0mm}{5mm} \hspace*{-1.6mm} \extext{#4}}}%
	\mbox{\framebox[8mm]{\rule[-1mm]{0mm}{5mm} \hspace*{-1.6mm} \extext{#5}}}%
	\mbox{\framebox[8mm]{\rule[-1mm]{0mm}{5mm} \hspace*{-1.6mm} \extext{#6}}}%
	\mbox{ \makebox[3mm]{\rule[-1mm]{0mm}{5mm} \hspace*{-2mm} .}}%
	\mbox{\framebox[8mm]{\rule[-1mm]{0mm}{5mm} \hspace*{-1.6mm} \extext{#7}}}%
	\mbox{\framebox[8mm]{\rule[-1mm]{0mm}{5mm} \hspace*{-1.6mm} \extext{#8}}}%
	\mbox{\framebox[8mm]{\rule[-1mm]{0mm}{5mm} \hspace*{-1.6mm} \extext{#9}}}%
}
\newcommand{\exstring}[1]{%
	\mbox{\framebox[0.9\textwidth][l]{\rule[-1mm]{0mm}{5mm} \hspace*{-1.6mm} \extext{#1}} \hspace*{2mm}}%
}


%%----------------------------------------------------------------------------
%% Info over het examen, voor FBO
%%----------------------------------------------------------------------------

\newcommand{\academiejaar}{2019 - 2020}        % vb. 2012-2013
\newcommand{\examenperiode}{$1^e$}              % vb. 1e, 2e
\newcommand{\faculteit}{Bedrijf en Organisatie}
\newcommand{\examendatum}{TODO}
\newcommand{\examenuur}{TODO}

\newcommand{\opleiding}{Toegepaste informatica, 2TI}
\newcommand{\olod}{Onderzoekstechnieken}       % vb. Algoritmen
\newcommand{\dolod}{Theorie}
\newcommand{\reeks}{examen}                 % vb. Reeks 1, Reeks 2, Inhaalexamen
\newcommand{\campus}{Aalst, ev. Schoonmeersen,}   % vb. Schoonmeerssen, Schoonmeersen
\newcommand{\lectoren}{Jens Buysse}   



%% new commands
\makeatletter
\newcommand{\ID}[1]{\def\@ID{#1}}
\newcommand{\Date}[1]{\def\@Date{#1}}
\ID{00001}
\Date{DD-MM-YYYY}

%% \exinput{header}

\newcommand{\myID}{\@ID}
\newcommand{\myDate}{\@Date}
\makeatother



\begin{document}
	\begin{tabulary}{\textwidth}{|L|L|}
		\hline
		\multicolumn{2}{|p{\textwidth}|}{\textbf{Academiejaar \academiejaar{} -- \examenperiode{} examenperiode \hfill \reeks}} \\ 
		\hline
		Faculteit: \faculteit                            & Examendatum: \\
		Opleiding, afstudeerrichting en jaar: \opleiding & \examendatum \\
		Naam van het opleidingsonderdeel: \olod          &  \\
		dOLOD/Deelexamen: \dolod                         & Aanvangsuur examen: \\
		Campus: \campus                                  & \examenuur \\
		Lesgever(s): \lectoren                           &  \\
		\hline
		\multicolumn{2}{|p{\textwidth}|}{\textbf{Voornaam en naam student:}} \\
		\hline
		\multicolumn{2}{|p{\textwidth}|}{Studentennummer:} \\
		\hline
		
		Lector bij wie de student de onderwijsactiviteit volgde: & Lesgroep v/d onderwijsactiviteit: \\
		& \\
		\hline
		\multicolumn{2}{|p{\textwidth}|}{\textbf{Behaald resultaat: \_\_\_\_\_ op \numpoints{}}} \\
		\hline
	\end{tabulary}

	\begin{itemize}
		\item Tijdens het oefeningenexamen mogen volgende hulpmiddelen gebruikt worden:
		\begin{itemize}
			\item Eigen laptop: R, Excel (of andere rekenbladsoftware), rekenmachine, internetverbinding;
			\item Afgedrukte cursus, eigen uitgewerkte oefeningen en nota's
		\end{itemize}
		\item Wanneer de uitkomst een reëel getal is, rond dan af tot \textbf{drie cijfers na de komma}.
	\end{itemize}
	
	Algemene richtlijnen:
	
	\begin{itemize}
		\item Vul het bovenstaande kader in. Vul op elke bladzijde je naam en voornaam in.
		\item Het laatste blad is leeg en kan dienen als kladpapier. Maak dit blad of de bundel zelf niet los!
		\item Voor studenten met Individuele Onderwijs- en ExamenMaatregelen: schrijf IOEM op elke bladzijde.
		\item Controleer of deze examenbundel alle pagina’s bevat, zo niet verwittig de docent of de toezichter zodat je een nieuw exemplaar krijgt.
		\item Je mag geen enkele vorm van communicatie  -ook niet draadloos of online- gebruiken tijdens de examens (chatten, mailen, Messenger, \ldots). GSM’s en dergelijke moeten \textbf{uitgeschakeld} zijn (niet op stand-by, trillen, \ldots). GSM’s, smartphones, smartwatches enz. mogen tijdens de examens ook NIET gebruikt worden om de tijd te raadplegen. Het niet volgen van deze gedragscode wordt gesanctioneerd als ``onregelmatigheden bij een examen'' (artikel 53 van OER).
	\end{itemize}
	
	Veel succes!

	
	\newpage
	
	\section{Antwoordenblad}
	%% \exinput{questionnaire}
	
	\newpage
	
	\begin{enumerate}
	
	\section{Vragen}
	%% \exinput{exercises}
		
	\end{enumerate}
	
\end{document}
